\documentclass[t,compress,aspectratio=169]{beamer}
\include{preamble}

\title{SPDX \& REUSE}
\subtitle{For better license statements}
\author{Andreas Cord-Landwehr}
\newcommand{\authoremail}{cordlandwehr@kde.org}

% Pick a page header image (alternatively create another square image and set that)
% \pgfdeclareimage[height=.1\paperheight]{pageheader}{categories/cassette.jpg}
% \pgfdeclareimage[height=.1\paperheight]{pageheader}{categories/cursor.jpg}
% \pgfdeclareimage[height=.1\paperheight]{pageheader}{categories/desktop.jpg}
% \pgfdeclareimage[height=.1\paperheight]{pageheader}{categories/disc.jpg}
% \pgfdeclareimage[height=.1\paperheight]{pageheader}{categories/gears.jpg}
\pgfdeclareimage[height=.1\paperheight]{pageheader}{categories/handshake.jpg}
% \pgfdeclareimage[height=.1\paperheight]{pageheader}{categories/headphones.jpg}
% \pgfdeclareimage[height=.1\paperheight]{pageheader}{categories/mouse.jpg}
% \pgfdeclareimage[height=.1\paperheight]{pageheader}{categories/thumbsup.jpg}
% \pgfdeclareimage[height=.1\paperheight]{pageheader}{categories/touch.jpg}

\lstset{ %
  language=C++,
%   backgroundcolor=\color{KDEgray4},
%   basicstyle=\footnotesize\ttfamily,
%   breakatwhitespace=false,
%   breaklines=true,
%   captionpos=b,
%   commentstyle=\color{KDEgreen},
%   escapeinside={\%*}{*)},
%   extendedchars=true,
%   frame=single,
%   keywordstyle=\color{KDEblue},
%   language=Prolog,
%   numbers=left,
%   numbersep=5pt,
%   numberstyle=\tiny\color{lightgray},
%   rulecolor=\color{lightgray},
%   showspaces=false,
%   showstringspaces=false,
%   showtabs=false,
%   stepnumber=1,
%   stringstyle=\color{KDEorange},
%   tabsize=2,
%   title=\lstname,
  morekeywords={Item,import,not,\},\{,Q_SIGNALS,public,Q_OBJECT,virtual,NOTIFY,Q_NULLPTR,Q_DISABLE_COPY,Q_DECL_OVERRIDE},
%   deletekeywords={time}
}

\begin{document}

\usebackgroundtemplate{\includegraphics[height=\paperheight]{1920x1080-akademy}}
\begin{withoutheadline}
\begin{frame}
\titlepage
\end{frame}
\end{withoutheadline}
\usebackgroundtemplate{\includegraphics[height=\paperheight]{1920x1080-noakademy}}

%==============================================================================

\begin{frame}
    {What is Copyright?}

    \begin{enumerate}
        \item A legal construct that grants someone exclusive rights over a creative work.
        \item Copyright is the right to ``make copies'' and give them away.
        \item Per default you have copyright over your work.
        \item Copyright can be transfered, e.g. to the KDE e.V.\\ (go to Ade's talk to see when this might be reasonable!)
    \end{enumerate}
    \bigskip

    More about copyrights:\\
    \url{https://creativecommons.org/faq/\#what-is-copyright-and-why-does-it-matter}
\end{frame}

\begin{frame}
    {What is a License?}

    Via copyright, your work is not reusable by anybody else -- a license changes this

    \begin{description}
        \item [License] defines under which terms your software can be reused
        \item [Free Software License] must grant the following 4 rights:
            \begin{enumerate}
                \item Use
                \item Study
                \item Share
                \item Improve
            \end{enumerate}
        \item [Copyleft license] requires that same rights preserve in derivative works\\ (e.g. GPL, LGPL)
        \item [Permissive license] only minimal restrictions of 4 freedoms, but no requirements for derivative works (e.g. BSD, MIT)
    \end{description}
\end{frame}

\begin{frame}[fragile]
    {How do I grant a license?}

    \begin{example}[Traditional License Header -- do NOT do this anymore, please]
    \tiny
    \begin{verbatim}
    /*
        This file is part of Rocs.
        Copyright 2008-2011  Tomaz Canabrava <tomaz.canabrava@gmail.com>
        Copyright 2008       Ugo Sangiori <ugorox@gmail.com>
        Copyright 2010       Wagner Reck <wagner.reck@gmail.com>
        Copyright 2014       Andreas Cord-Landwehr <cordlandwehr@kde.org>

        This program is free software; you can redistribute it and/or
        modify it under the terms of the GNU General Public License as
        published by the Free Software Foundation; either version 2 of
        the License, or (at your option) any later version.

        This program is distributed in the hope that it will be useful,
        but WITHOUT ANY WARRANTY; without even the implied warranty of
        MERCHANTABILITY or FITNESS FOR A PARTICULAR PURPOSE.  See the
        GNU General Public License for more details.

        You should have received a copy of the GNU General Public License
        along with this program.  If not, see <https://www.gnu.org/licenses/>.
    */
    \end{verbatim}
    \end{example}
\end{frame}

\begin{frame}
    {Problems with those Statements}

    \begin{itemize}
        \item Long license texts are very error-prone and hard to check
        \item License statements are handcrafted (over 36 (!!!) different statements for LGPL-2.0-or-later in KF5)
            \begin{itemize}
                \item Update FSF address
                \item No automatic checking possible, or only with fuzzy checkers
                \item Often ambiguous handcrafted statements
            \end{itemize}
    \end{itemize}

    \textbf{Solution: SPDX Markers}
    \begin{itemize}
        \item SPDX license list provides unique license IDs: \url{https://spdx.org/licenses/}
        \item Standardized set of machine-readible expressions:
            \begin{itemize}
                \item \texttt{SPDX-FileCopyrightText: YEAR AUTHOR <CONTACT>}
                \item \texttt{SPDX-License-Identifier: LICENSE-IDENTIFIER}
            \end{itemize}
    \end{itemize}
\end{frame}

\pgfdeclareimage[height=.1\paperheight]{pageheader}{reuse.png}
\begin{frame}
    {REUSE -- to make it simple!}

    \begin{itemize}
        \item SPDX is a specification, written by legal experts:\\
            \url{https://spdx.github.io/spdx-spec/}
        \item \url{https://REUSE.software} is initiative by FSFE to make reusing easier
        \item Provide a simple specification that requires only tiny subset and gives guidelines how to apply it
        \item In a nutshell:
            \begin{enumerate}
                \item Add SPDX-License-Identifier tag to every file
                \item Add SPDX-FileCopyrightText tag to every file
                \item Add license text in \texttt{LICENSES/<license>.txt} for every license
            \end{enumerate}
        \item See KDE Licensing HowTo Wiki page: \url{https://community.kde.org/Guidelines_and_HOWTOs/Licensing}
    \end{itemize}
\end{frame}

\pgfdeclareimage[height=.1\paperheight]{pageheader}{categories/thumbsup.jpg}
\begin{frame}[fragile]
    {How do I grant a license, today?}

    \begin{example}[REUSE Compatible License Statement]
    \tiny
    \begin{verbatim}
    /*
        This file is part of Rocs.
        SPDX-FileCopyrightText: 2008-2011 Tomaz Canabrava <tomaz.canabrava@gmail.com>
        SPDX-FileCopyrightText: 2008 Ugo Sangiori <ugorox@gmail.com>
        SPDX-FileCopyrightText: 2010 Wagner Reck <wagner.reck@gmail.com>
        SPDX-FileCopyrightText: 2014 Andreas Cord-Landwehr <cordlandwehr@kde.org>

        SPDX-License-Identifier: GPL-2.0-or-later
    */
    \end{verbatim}
    \end{example}
\end{frame}

\begin{frame}
    {Why does it matter?}

    \begin{itemize}
        \item Not every license is compatible with every other license :-/\\
            $\rightarrow$ application/library \textbf{cannot} be shipped to users if conflicting
        \item We have a policy that strives for compatible licenses:\\
            \url{https://community.kde.org/Policies/Licensing_Policy}\\
            \begin{description}
                \item [Applications:] GPL
                \item [Libraries:] LGPL
                \item [Build System:] BSD
                \item [Resources:] CC
                \item [Documentation:] CC (used to be FDL)
            \end{description}
        \item For GPL/LGPL licenses devil lies in the version number details
    \end{itemize}
\end{frame}

\begin{frame}[fragile]
    {A Glimpse of Better Tooling}

    Still quite tedious work to check if we do something legally right, but now\dots
    \begin{itemize}
        \item Get the REUSE tool: \texttt{pip install reuse}\\
            $\rightarrow$ tells you if you are fully REUSE compliant
        \item License compatibility test generator soon in extra-cmake-modules:\\
            $\rightarrow$ \url{https://invent.kde.org/frameworks/extra-cmake-modules/-/merge_requests/21}
    \end{itemize}

    \begin{example}[Check source compatibility with outbound license]
    \tiny
    \begin{verbatim}
    include(ECMCheckOutboundLicense)

    file(GLOB TEST_FILES "*.cpp" "*.h")
    ecm_check_outbound_license(
        LICENSES LGPL-2.1-only LGPL-3.0-only
        TEST_NAME mylibrary
        FILES ${TEST_FILES}
    )
    \end{verbatim}
    \end{example}
\end{frame}

\begin{frame}
    \frametitle{The End}
    \vspace{0.3cm}
    \begin{block}{}
        \centering\begin{Huge}Convert your project!\end{Huge}
    \end{block}
    \medskip

    \begin{itemize}
        \item KDE Frameworks is done! ($\sim$7500 files)
        \item KDE PIM is done, too!
        \item \dots and your project next?
    \end{itemize}

    \vspace{0.5cm}
    \begin{tiny}
        \begin{block}{Easy Steps to Follow}
            \begin{description}
                \item [KDE License HowTo] \url{https://community.kde.org/Guidelines_and_HOWTOs/Licensing}
                \item [Licensedigger] KDE conversion tooling: \url{https://invent.kde.org/sdk/licensedigger}
                \item [Support] irc: CoLa, mail: \url{cordlandwehr@kde.org}
            \end{description}
        \end{block}
    \end{tiny}
\end{frame}

\end{document}
